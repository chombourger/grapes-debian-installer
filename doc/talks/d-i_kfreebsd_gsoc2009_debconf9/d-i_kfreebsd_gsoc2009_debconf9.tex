\documentclass{beamer}


\mode<presentation>
{
  \usetheme{Warsaw}
  % or ...

  \setbeamercovered{transparent}
  % or whatever (possibly just delete it)
}


\usepackage[english]{babel}
% or whatever

\usepackage{pdfpages}
\usepackage{soul}

\usepackage{ucs}
\usepackage[utf8x]{inputenc}
% or whatever

\usepackage{times}
\usepackage[T1]{fontenc}


\title
{Debian-Installer Support for GNU/kFreeBSD}

\author
{Luca ~Favatella}
% - Give the names in the same order as the appear in the paper.
% - Use the \inst{?} command only if the authors have different
%   affiliation.

% - Use the \inst command only if there are several affiliations.
% - Keep it simple, no one is interested in your street address.

\date[Debconf 9] % (optional, should be abbreviation of conference name)
{DebConf 9, Cáceres,\\ Extremadura,\\ Spain}
% - Either use conference name or its abbreviation.
% - Not really informative to the audience, more for people (including
% yourself) who are reading the slides online




\pgfdeclareimage[height=1cm]{debian-logo}{debian-swirl}
\logo{\pgfuseimage{debian-logo}}



% Delete this, if you do not want the table of contents to pop up at
% the beginning of each subsection:
%\AtBeginSection[]
%{
%  \begin{frame}<beamer>
%    \frametitle{Outline}
%    \tableofcontents[currentsection]
%  \end{frame}
%}


% Delete this, if you do not want the table of contents to pop up at
% the beginning of each subsection:
%\AtBeginSubsection[]
%{
%  \begin{frame}<beamer>
%    \frametitle{Outline}
%    \tableofcontents[currentsection,currentsubsection]
%  \end{frame}
%}


% If you wish to uncover everything in a step-wise fashion, uncomment
% the following command: 
%\beamerdefaultoverlayspecification{<+->}


\begin{document}

\begin{frame}
  \titlepage
\end{frame}

%\begin{frame}
%  \tableofcontents
%\end{frame}

%%%%%%%%%%%%%%%%%

\begin{frame}
  \frametitle{About}

  GNU/kFreeBSD is currently using a hacked version of the FreeBSD installer combined with crosshurd as its own installer. While this works more or less correctly for standard installations (read: the exact same installation as in the documentation), it does not allow any changes in the installation process except the hard disk partitioning. This project is about porting debian-installer on GNU/kFreeBSD, and to a bigger extent, make debian-installer less Linux dependant.

  \begin{itemize}
  \item
    GSoC 2009
    \begin{itemize}
    \item
      student: Luca Favatella, University of Catania (Italy)
    \item
      mentor: Aurelien Jarno 
    \end{itemize}
  \item
    previous efforts by Robert Millan in 2006 [0]
  \end{itemize}

\end{frame}

\begin{frame}
  \frametitle{TODO (in the GSoC 2009 period)}

  \begin{itemize}
  \item
    \st{look for important missing udebs}
  \item
    BusyBox udeb
    \begin{itemize}
    \item
      \st{enable needed not network related config options} [1]
    \item
      enable some network related config options (if needed)
    \item
      fix ash job control bug (workarounded at the moment)
    \end{itemize}
  \item
    \st{d-i config files} (thanks to Robert Millan)
  \item
    \st{kernel udeb} (thanks to Robert Millan)
  \item
    \st{rootskel package} (thanks to Robert Millan)
  \item
    port and enable d-i components to resemble the config options of the current GNU/kFreeBSD installer [2]
    \begin{itemize}
    \item
      keyboard layout selection
    \item
      network configuration
    \item
      udebs download and installation
    \item
      disk partitioning
    \end{itemize}
  \end{itemize}

\end{frame}

\begin{frame}
  \frametitle{How you can help}

  \begin{itemize}
  \item
    debian-installer
    \begin{itemize}
    \item
      split building of rootskel-bootfloppy in rootskel
    \item
      port parted to GNU/kFreeBSD
    \item
      port console-setup to GNU/kFreeBSD
    \item
      kfreebsd-i386 kernel udeb
      \begin{itemize}
      \item
        take a look at massbuild [3]
      \item
        build it and analyze console output
      \end{itemize}
    \item
      read code and make it less Linux specific
    \end{itemize}
  \item
    BusyBox
    \begin{itemize}
    \item
      add Linux compatible sysinfo() to GNU/kFreeBSD
    \item
      fix ash job control bug
    \item
      enable network related config options already present on GNU/Linux udeb
    \item
      review patches at [1] and get them merged upstream
    \item
      read code and make it less Linux specific
    \end{itemize}
  \end{itemize}

  \begin{itemize}
  \item
    please ask on debian-boot@l.d.o before starting
  \end{itemize}

\end{frame}

\begin{frame}
  \frametitle{References}

  \begin{itemize}
  \item
    \text{[0]}
    http://slackydeb.blogspot.com/2009/04/previous-attempts-to-port-d-i-to\_27.html
  \item
    \text{[1]}
    http://slackydeb.blogspot.com/2009/05/busybox-status-on-gnukfreebsd.html
  \item
    \text{[2]}
    http://glibc-bsd.alioth.debian.org/doc/installing.html
  \item
    \text{[3]}
    http://lists.debian.org/debian-boot/2009/06/msg00687.html
  \end{itemize}

  \begin{itemize}
  \item
    project updates
    \begin{itemize}
    \item
      progress reports: soc-coordination@alioth.debian.org
    \item
      vcs: svn://svn.debian.org/svn/d-i/branch/d-i/kfreebsd/
    \item
      blog (news \& some code): http://slackydeb.blogspot.com
    \end{itemize}
  \item
    irc: \#debian-boot, \#debian-kbsd
  \item
    mailing lists: debian-boot@l.d.o, debian-bsd@l.d.o
  \end{itemize}

\end{frame}

\end{document}
