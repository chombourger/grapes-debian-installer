\documentclass{beamer}


\mode<presentation>
{
  \usetheme{Warsaw}
  % or ...

  \setbeamercovered{transparent}
  % or whatever (possibly just delete it)
}


\usepackage[english]{babel}
% or whatever

\usepackage{pdfpages}

\usepackage{ucs}
\usepackage[utf8x]{inputenc}
% or whatever

\usepackage{times}
\usepackage[T1]{fontenc}


\title[Some bits about the Debian Installer] % (optional, use only with long paper titles)
{Some bits about the Debian Installer}

\author[joeyh, h01ger, bubulle, fjp] % (optional, use only with lots of authors)
{Joey ~Hess, Holger ~Levsen, Christian ~Perrier, Frans ~Pop}
% - Give the names in the same order as the appear in the paper.
% - Use the \inst{?} command only if the authors have different
%   affiliation.

% - Use the \inst command only if there are several affiliations.
% - Keep it simple, no one is interested in your street address.

\date[Debconf 5] % (optional, should be abbreviation of conference name)
{5th Debian Conference, Helsinki,\\ Finland}
% - Either use conference name or its abbreviation.
% - Not really informative to the audience, more for people (including
% yourself) who are reading the slides online




\pgfdeclareimage[height=2cm]{debian-logo}{debian-swirl}
\logo{\pgfuseimage{debian-logo}}



% Delete this, if you do not want the table of contents to pop up at
% the beginning of each subsection:
\AtBeginSection[]
{
  \begin{frame}<beamer>
    \frametitle{Outline}
    \tableofcontents[currentsection]
  \end{frame}
}


% Delete this, if you do not want the table of contents to pop up at
% the beginning of each subsection:
%\AtBeginSubsection[]
%{
%  \begin{frame}<beamer>
%    \frametitle{Outline}
%    \tableofcontents[currentsection,currentsubsection]
%  \end{frame}
%}


% If you wish to uncover everything in a step-wise fashion, uncomment
% the following command: 
%\beamerdefaultoverlayspecification{<+->}


\begin{document}

\begin{frame}
  \titlepage
\end{frame}

\begin{frame}
  \tableofcontents
\end{frame}

%%%%%%%%%%%%%%%%%

\section{Past and present}

\begin{frame}
  \frametitle{R.I.P: boot-floppies}
	\begin{itemize}
	\item
	 	Ahead of the times in the early nineties
	\item
		In maintenance mode since the end of the nineties
	\item
		A pain already for Woody: 
		\begin{itemize}
			\item
				Lots of new architectures
			\item
				Localization 
			\item
				Designed for floppies, not for CD or netboot
		\end{itemize}
	\end{itemize}
\end{frame}

\begin{frame}
  \frametitle{d-i design and key-features}
	\begin{itemize}
	\item
		debconf 
		\begin{itemize}
		\item
			Different frontends are possible
		\item
			I18n
		\item
			Preseeding files for automation
		
		\end{itemize}
	\item
		udebs
		\begin{itemize}
		\item
			Modularisation
		\item
			Small packages that fit better into ramdisk
		\item
			Don't have to comply with policy
		\end{itemize}
	\item
		main-menu	
		\begin{itemize}
		\item
			Dynamically orders udebs by dependencies
		\item
			Can be hidden from the user (by DEBCONF\_PRIORITY)			
		\end{itemize}
	\end{itemize}
\end{frame}

\begin{frame}
  \frametitle{d-i development process}
	\begin{itemize}
	\item
		Team work, different people concentrate on specific areas, >100 committers
	\item
		Informal leadership, decisions are based on consensus or who does it first and well
	\item
		Coordinated over mailing-list debian-boot@l.d.o, IRC channel \#debian-boot on freenode \& RL meetings 
	\item
		Subversion repository at svn+ssh://svn.d-i.alioth.debian.org/svn/d-i/
		\begin{itemize}
			\item
				Commit messages go to mailinglist and IRC
			\item
				Trunk for new developments
			\item
				People branches for experimental stuff
			\item
				Sarge branch
		\end{itemize}
	\end{itemize}
\end{frame}


%%%%%%%%%%%%%%%%%

\section{Debian Installer internationalization and localization}

\begin{frame}
  \frametitle{Drawing ideas}
	\begin{itemize}
	\item
		English sucks
	\item
		All displayed texts must be translated
	\item
		Use of debconf
	\item
		Use of gettext (po-debconf)
	\item
		All translatable material to debconf templates
	\end{itemize}
\end{frame}

\begin{frame}
  \frametitle{Towards World Domination: potato}
  \pgfdeclareimage[width=10cm]{potato}{potato}
  \pgfuseimage{potato}
\end{frame}

\begin{frame}
  \frametitle{Towards World Domination: woody}
  \pgfdeclareimage[width=10cm]{woody}{woody}
  \pgfuseimage{woody}
\end{frame}

\begin{frame}
  \frametitle{Towards World Domination: sarge}
  \pgfdeclareimage[width=10cm]{sarge}{sarge}
  \pgfuseimage{sarge}
\end{frame}

\begin{frame}
  \frametitle{Towards World Domination: etch}
  \pgfdeclareimage[width=10cm]{etch}{etch}
  \pgfuseimage{etch}
\end{frame}

\begin{frame}
  \frametitle{Levels: from d-i translation to World Domination}
	\begin{itemize}
	\item
		Level 1: ``core'' d-i packages (stage 1)
	\item
		Level 2: default base system install - configuration
	\item
		Level 3: any level of base system install - configuration
	\item
		Level 4: any level of base system install - messages
	\item
		Level 5: default desktop install
	\item
		Level 6: debconf templates for all standard packages
	\item
		Level 7: Debian fully translated
	\end{itemize}
\end{frame}

\begin{frame}
  \frametitle{The future: Thou shalt be translated?}
	\begin{itemize}
	\item
		New languages: assist new translators
		\begin{itemize}
			\item
				More locales
			\item
				Easier to use framework (Rosetta, Pootle, DDTP?)
		\end{itemize}
	\item
		New languages: we need the graphical installer
		\begin{itemize}
			\item
				Combining languages
			\item
				Stop depending on Unicode ``text-mode'' fonts
			\item
				Better RTL and Bidi display
			\item
				Klingon support and Universe domination
		\end{itemize}
	\end{itemize}
\end{frame}

%%%%%%%%%%%%%%%%%

\section{The Installation Guide}

\begin{frame}
  \frametitle{Introduction}
	\begin{itemize}
	\item
		Overview
	\item
		Status
	\item
		Translations
	\item
		Future
	\item
		Technical aspects: see our paper
	\end{itemize}
\end{frame}

\begin{frame}
  \frametitle{Overview}
	\begin{itemize}
	\item
		One source for all architectures
	\item
		Available as: HTML, PDF, text, (Postscript)
	\item
		Available from several sources
		\begin{itemize}
		\item
			official website
%			updated!
		\item
			1st full CD/DVD
		\item
			package debian-installer-manual
		\item
			http://d-i.alioth.debian.org/manual/ (development version)
		\end{itemize}
	\item
		Daily builds create manual ``for Etch''
	\end{itemize}
\end{frame}

\begin{frame}
  \frametitle{Status}
	\begin{itemize}
	\item
		Still largely based on manual for Woody :-(
%		(d-i components documented OK)
	\item
		Not well balanced
	\item
		Need help: arm, hppa, mips, mipsel, powerpc, s/390
	\end{itemize}
  \pgfdeclareimage[width=10cm]{manual-warning}{manual-warning}
  \pgfuseimage{manual-warning}
\end{frame}

\begin{frame}
  \frametitle{Translations}
	\begin{itemize}
	\item
		Originally direct translation of XML files: 7/12 languages
	\item
		Last year support for PO files was added: 13/19 languages
	\item
		Besides English, fully translated into 11 languages:
		\begin{itemize}
		\item
			French, German, Spanish, Portuguese (pt and pt\_BR)
		\item
			Japanese, Chinese (zh\_CN and zh\_TW), Korean
		\item
			Russian, Czech
		\end{itemize}
	\end{itemize}
\end{frame}

\begin{frame}
  \frametitle{Future}
	\begin{itemize}
	\item
		More translations
	\item
		Better PDF support (oriental languages, Russian, Greek)
%		Initial work done by Nikolai Prokoschenko using db2latex
	\item
		Better integration with DDP
%		Includes improving/packaging build system
	\item
		Restructure/split manual?
	\end{itemize}
    \vspace{1cm}
	
	\huge{Your help is welcome!}
\end{frame}

%%%%%%%%%%%%%%%%%

\section{The future of Debian Installer: how to help}

\begin{frame}
  \frametitle{Up and to the right?}
  \pgfdeclareimage[width=9.5cm]{svnplot}{svngraph/svnplot}
  \pgfuseimage{svnplot}
\end{frame}

\begin{frame}
  \frametitle{TODO}
  	\begin{itemize}
	\item UTF-8 default
	\item d-i updates for sarge
	\item support installing sarge using etch installer
	\item graphical installer
	\item improve automated installs
	\item encrypted filesystem support
	\item improve hardware detection (hotplug, sata, nic renames)
	\item improve support for non-free firmware and drivers
	\item accessability for the blind
	\item installs via PPPOE
	\item move base-config into first stage install
	\item disk space size checking for tasks
	\item make it easier to customise d-i
	\item remove a single question from the standard install
	\end{itemize}
\end{frame}

\end{document}
